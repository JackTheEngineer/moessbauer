\documentclass[11pt,a4paper]{scrartcl}
\usepackage[utf8]{inputenc}
\usepackage{geometry}
\geometry{a4paper, top=25mm, left=25mm, right=25mm, bottom=30mm, headsep=10mm, footskip=12mm}
\usepackage[ngerman]{babel}
\usepackage[T1]{fontenc}
\usepackage{amsmath}
\usepackage{physics}
\usepackage{mathtools}
\usepackage{xfrac}
\usepackage{amsfonts}
\usepackage{amssymb}
\usepackage{graphicx}
\usepackage{float}
\usepackage{url}
\usepackage{subfigure}
\usepackage{textcomp}
\usepackage{ngerman}
\usepackage{verbatim}
\usepackage{slashed}
\usepackage{bbm}
\usepackage{tikz}
\title{Der Mössbauer Effekt}
\author{Gruppe 26\\ Anh Tong \\ Tobias Theil \\ Kholodkov Jakov }
\date{\today}

% stands for the degree sign 
\newcommand{\dee}{{$^{\circ}$}} 	
\newcommand{\degr}{^{\circ}}
\newcommand{\shf}{$SF_6$}
% \newcommand{\abs}[1]{\ensuremath{\left\vert#1\right\vert}}

\begin{document}
	
\maketitle
\tableofcontents
\bibliography{literatur}
\bibliographystyle{unsrt}

\newpage
\section{Einleitung}

%%% Local Variables:
%%% mode: latex
%%% TeX-master: "../motors.tex"
%%% End:
\section{Methoden}


%%% Local Variables:
%%% mode: latex
%%% TeX-master: "../Laser"
%%% End:

\section{Experimentelles Vorgehen}

%%% Local Variables:
%%% mode: latex
%%% TeX-master: "../motors"
%%% End:

\section{Auswertung}

%%% Local Variables:
%%% mode: latex
%%% TeX-master: "../Moessbauer"
%%% End:

\section{Berechnung der Messunsicherheiten}

%%% Local Variables:
%%% mode: latex
%%% TeX-master: "../Moessbauer"
%%% End:

\section{Fragen}

%%% Local Variables:
%%% mode: latex
%%% TeX-master: "../Moessbauer"
%%% End:


% Anhaege 
\begin{appendix}
\section{Anhang}


\end{appendix}
 
\end{document}

%%% Local Variables:
%%% mode: latex
%%% TeX-master: t
%%% End:

